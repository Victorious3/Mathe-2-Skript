\ifdefined\MAINDOC\else
\documentclass[10pt, a4paper, fleqn]{article}
\usepackage{base}

\begin{document}
    \title{Skript Mathe 2}
    \date{18. Juni 2018}
    \maketitle
\fi
\textbf{Beweis: } $\boxed{1. \Rightarrow 2.}$ Sei $f$ in $x_0$ differenzierbar.
\[\begin{aligned}
    &\text{Setze } R(x) = \begin{cases}
        \dfrac{f(x) - f(x_0)}{x - x_0} - f'(x_0) & x \neq x_0 \\
        0 & x = x_0
    \end{cases} \\
    &\imp \lim_{n \to 0} R(x_0 + h) = \lim_{n \to 0} \frac{f(x_0 + h) - f(x_0)}{h} - f'(x_0) \\
    &= R(x_0) = 0
\end{aligned}\]
$\imp R$ stetig in $x_0, R(x_0) = 0$ und (*) ist erfüllt für $m = f'(x_0)$.
\bigskip

$\boxed{2. \Rightarrow 1.}$ Gelte (*) für ein $m \in \IR$ und eine in $x_0$ stetige Funktion 

$R: I \to \IR, R(x_0) = 0$.
\[\begin{aligned}
    &f(x_0 + h) = f(x_0) + m \cdot h + R(x_0 + h) \cdot h \\
    &\underset{h \neq 0}{\eqv} \frac{f(x_0 + h) - f(x_0)}{h} = m + R(x_0 + h) \\
    &\xrightarrow[h \to 0]{} \lim_{h \to 0} \frac{f(x_0 + h) - f(x_0)}{h} = m + \underbrace{R(x_0)}_{= 0} \\
\end{aligned}\]
da $R(x_0) = 0$ und stetig in $x_0 \qed$

\subsection{Satz}
Wenn $f$ differenzierbar in $x_0 \in I \imp f$ stetig.

\textbf{Beweis: } Folge aus 6.5/2 (*), da $f$ Summe in $x_0$ stetiger Funktionen.

\subsection{Bemerkung}
Die Umkehrung von 6.6 gilt nicht. In $x_0 = 0$ hat $f'(x_0) = |x|$ einen Knick:
\begin{itemize}
    \item $\lim\limits_{n \to 0^-} \dfrac{|0 + h| - h}{h} = \lim\limits_{h \to 0} \dfrac{-h}{h} = -1$
    \item $\lim\limits_{n \to 0^+} \dfrac{|0 + h| - h}{h} = \lim\limits_{h \to 0} \dfrac{h}{h} = 1$
\end{itemize}
$\underset{5.12}{\imp}$ In $x_0 = 0$ existiert keine Ableitung.

\section*{Rechenregeln}

\subsection{Satz: Ableitungsregeln}
$f, g: I \to \IR$ differenzierbar in $x \in I$.

Dann sind auch $c \cdot f$ (für $c \in \IR$), $f \pm g, f \cdot g$ und \\
$\frac{f}{g}$ (für $g(x) \neq 0$) differenzierbar in x mit:
\begin{enumerate}[a)]
    \item $(c \cdot f)'(x) = c \cdot f'(x)$
    \item $(f \pm g)'(x) = f'(x) \pm g'(x)$
    \item Produktregel:
    
    $(f \cdot g)'(x) = f'(x) \cdot g(x) + f(x) \cdot g'(x)$
    \item Quotientenregel:

    $\qt{\frac{f}{g}}'(x) = \dfrac{f'(x) \cdot g(x) - f(x) \cdot g'(x)}{g^2(x)}$
\end{enumerate}

\textbf{Beweis: }
\begin{enumerate}
    \item[a, b)] Übung
    \item[c)]
    \[\begin{aligned}
        &\frac{(fg)(x + h) - (fg)(x)}{h} \\
        &= \frac{f(x + h)g(x + h) \overbrace{-f(x)g(x + h) + f(x)g(x + h)}^{= 0} - f(x)g(x)}{h} \\
        &= \frac{(f(x + h)- f(x)) \cdot g(x + h)}{h} + \frac{f(x) \cdot (g(x + h) - g(x))}{h} \\
        &\xrightarrow[h \to 0]{} f'(x) \cdot g(x) + f'(x) \cdot g'(x) \quad \text{ (da $g$ stetig)}
    \end{aligned}\]
    \item[d)]
    \[
        \frac{\qt{\frac{f}{g}}(x + h) - \qt{\frac{f}{g}}(x)}{h} =
        \frac{f(x+h)g(x) - f(x) - g(x + h)}{h \cdot g(x + h) \cdot g(x)}
    \]
    Schiebe wie in c) im Zähler $-f(x + h) g(x + h) + f(x + h) g(x + h)$ ein und erhalte
    mit $h \to 0$ die Behauptung. $\qed$
\end{enumerate}

\subsection{Beispiele}
\begin{enumerate}[a)]
    \item Wegen 6.8a,d) ist jedes Polynom und jede rationale Funktion differenzierbar.
    \item $(4x^3 + 7x + 5)' = 12x^2 + 7$
    \item $\qt{\dfrac{\sin x}{x}}' = \dfrac{\cos x \cdot x - \sin x}{x^2} \quad (x \neq 0)$
    \item $(\tan x)' = \dfrac{\cos^2 x + \sin^2 x}{\cos^2 x} = \dfrac{1}{\cos^2 x}$
\end{enumerate}

\subsection{Satz: Kettenregel}
Die Verknüpfung $f \circ g$ zweier differenzierbarer Funktionen $f, g$ ist differenzierbar und
es gibt $(f \circ g)' = (f' \circ g) \cdot g'$ bzw $\frac{d}{dx} f(g(x)) = f'(g(x)) \cdot g'(x)$.

\textbf{Beweis: } Mit Substitution:

$\tilde{x} = g(x), \ \tilde{h} = g(x + h) - g(x)$

Es gilt: $h \to 0 \imp \tilde{h} \to 0$ da $g$ stetig. Damit ist
\[\begin{aligned}
    &\frac{f(g(x + h)) - f(g(x))}{h} \\
    &= \frac{f(g(x + h) - g(x) + g(x)) - f(g(x))}{g(x + h) - g(x)} \cdot \frac{g(x + h) - g(x)}{h} \\
    &= \frac{f(\tilde{x} + \tilde{h}) - f(\tilde{x})}{\tilde{h}} \cdot \frac{g(x + h) - g(x)}{h} \\
    &\xrightarrow[h \to 0]{} f'(\tilde{x}) \cdot g'(x) = f'(g(x)) \cdot g'(x) \qed
\end{aligned}\]

\subsection{Beispiel}
\[
    (\overbrace{\sin (\underbrace{5x^2}_{g})}^{f \circ g})' = \underbrace{10x}_{g'} \underbrace{\cdot \cos(5x^2)}_{f' \circ g}
\]

\subsection{Veranschaulichung zur Ableitung der Umkehrfunktion}
\begin{tikzpicture}[
    declare function = {
        f(\x) = 3.289562 + 0.6189274*\x + 0.09662097*\x^2 + 0.01076239*\x^3;
        t(\x) = 1.304*\x + 2.349;
        t2(\x) = -0.766871*(2.349 - \x);
    }
]
    \begin{axis}[
        width = \textwidth,
        height = \textwidth,
        axis x line = center,
        axis y line = center,
        xtick = {-7}, ytick = {-7},
        xticklabels = \empty, yticklabels = \empty,
        ymin = -12, ymax = 12,
        xmin = -12, xmax = 12,
        axis line style = {->},
        domain = -12:12,
        clip = false
    ]
    \addplot[color = black, domain = -11:7.4] {t(x)};
    \addplot[color = black] {t2(x)};
    \addplot[dashed, color = black] {x};
    
    % Not true, bad
    %\draw (-7, 0.3) -- (-8, 2) node[yshift = 4pt] {$m = 0$};
    %\draw (0.3, -7) -- (2, -8) node[right] {$m'$ unendlich groß};

    \draw (-12.5, -9.5) node {$f$};
    \draw (-7, -12.5) node {$f^{(-1)}$};

    \draw[->] (1.8, 10.5) to [out = 150, in = 60] (-4, 8) node[left] {$m = \dfrac{\Delta x}{\Delta y}$};
    \draw[->] (10.5, 1.8) to [out = -30, in = -170] (14, 1) node[right] {$m' = \dfrac{\Delta x}{\Delta y} = \dfrac{1}{m}$};

    \coordinate (A) at (2.5, {t(2.5)});
    \coordinate (B) at (2.5, {t(6)});
    \coordinate (C) at (6, {t(6)});

    \coordinate (D) at ({t(2.5)}, {t2(t(2.5))});
    \coordinate (E) at ({t(6)}, {t2(t(2.5))});
    \coordinate (F) at ({t(6)}, {t2(t(6))});

    \draw (A) node[circle, fill = black, scale = 0.3]{};
    \draw (D) node[circle, fill = black, scale = 0.3]{};

    \draw (A) -- (B) -- (C);
    \draw[decorate, decoration = {brace, amplitude = 10pt, raise = 2pt}] (A) -- (B) node[midway, left, xshift = -10pt] {$\Delta y$};
    \draw[decorate, decoration = {brace, amplitude = 10pt, raise = 2pt}] (B) -- (C) node[midway, above, yshift = 10pt] {$\Delta x$};

    \draw (D) -- (E) -- (F);
    \draw[decorate, decoration = {brace, amplitude = 10pt, raise = 2pt}] (F) -- (E) node[midway, right, xshift = 10pt] {$\Delta x$};
    \draw[decorate, decoration = {brace, amplitude = 10pt, raise = 2pt}] (E) -- (D) node[midway, below, yshift = -10pt] {$\Delta y$};


    \end{axis}

    % Second axis so that we can mirror without running into problems
    \begin{axis}[
        hide axis,
        width = \textwidth,
        height = \textwidth,
        ymin = -12, ymax = 12,
        xmin = -12, xmax = 12,
        domain = -12:12,
    ]
    \addplot[color = black, domain = -12:6] {f(x)};
    \addplot[rotate = -90, xscale = -1, color = black, domain = -12:6] {f(x)};
    \end{axis}

\end{tikzpicture}

\[\begin{aligned}
    &m = f'(x_0) \neq 0 \imp (f^{-1}(y_0))' = m' = \frac{1}{m} \\
    &= \frac{1}{f'(x_0)} = \frac{1}{f'(f^{-1} (y_0))}
\end{aligned}\]

\subsection{Satz: Ableitung der Umkehrfunktion}
$I, J$ offene Intervalle, $f: I \to J$ differenzierbar in $x_0 \in I$ mit
$f'(x_0) \neq 0$. Dann:

$f^{-1}: y \to I$ differenzierbar in $y_0 = f(x_0)$ mit $(f^{-1}(y_0))' = \dfrac{1}{f'(x_0)} = \dfrac{1}{f'(f^{-1}(x_0))}$

\textbf{Beweis: } Sei $t = f(x_0 + h) - f(x_0) \quad (*)$

Es gilt: $h \to 0 \eqv t \to 0$
\[\begin{aligned}
    &\frac{1}{\dfrac{f(x_0 + h) - f(x_0)}{h}} \overset{(*)}{=} \frac{x_0 + h - x_0}{t} \\
    & = \frac{f^{-1}(f(x_0 + h)) - f^{-1}(f(x_0))}{t} \overset{(*)}{=} \frac{f^{-1}(f(x) + t) - f^{-1}(f(x))}{t} \\
    &\xrightarrow[h \to 0]{} \frac{1}{f'(x_0)} = \frac{1}{f'(f^{-1}(y_0))} \qed
\end{aligned}\]

\ifdefined\MAINDOC\else
\end{document}
\fi