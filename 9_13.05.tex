\ifdefined\MAINDOC\else
\documentclass[10pt, a4paper, fleqn]{article}
\usepackage{base}

\begin{document}
    \title{Skript Mathe 2}
    \date{13. Mai 2018}
    \maketitle
\fi
    \section*{Berechnung von Konvergenzradien}
    \subsection{Satz: Formel von Cauchy-Hademard}
    
    Sei $(a_k)_{k \geq 0}$ Folge in $\IR$ und $\displaystyle\lambda := \varlimsup_{k \to \infty} \sqrt[k]{|a_k|}$.
    $\rho$ sei der Konvergenzradius von $P(x) = \sum\limits_{k=0}^\infty a_k (x-x_0)^k$.
    
    Dann:
    $$
        \rho = \begin{cases}
            \frac{1}{\lambda} &\text{, falls } \lambda \in \IR > 0 \\
            0 &\text{, falls } \lambda = \infty \\
            \infty &\text{, falls } \lambda = 0
        \end{cases}
    $$

    \textbf{Beweis: }
    Wurzelkriterium: $\displaystyle\lambda := \varlimsup_{k \to \infty} \sqrt[k]{|a_k| \cdot |x-x_0|^k} = \lambda \cdot |x-x_0|$
    \[\begin{aligned}
        &\bullet \underbrace{\lambda \cdot |x-x_0}_{\mathclap{\text{D.h. } P(x) \text{ konvergiert}}} | < 1 \eqv |x-x_0| < \frac{1}{\lambda} \quad (= \rho) \\
        &\bullet \underbrace{\lambda \cdot |x-x_0}_{\mathclap{\text{D.h. } P(x) \text{ divergiert}}} | > 1 \eqv |x-x_0| > \frac{1}{\lambda} \quad (= \rho) \\
        &\imp \rho \text{ Konvergenzradius von } P(x)
    \end{aligned}\]

    \subsection{Beispiel}
    Für welche $x \in \IR$ ist $\sum\limits_{k=1}^\infty \frac{x^k}{k}$ konvergent?
    \[\begin{aligned}
        \bullet &\varlimsup_{k \to \infty} \sqrt[k]{\abs{\frac{1}{k}}} = \varlimsup_{k \to \infty} \frac{1}{\sqrt[k]{k}} = 1 = \lambda \\
        &\underset{3.8}{\imp} \rho = \frac{1}{\lambda} = 1 \\
        &\imp P(x) \text{ konvergent für } x \in \overbrace{(-1, 1)}^{\mathclap{x_0 - \rho, x_0 + \rho}} \text{ und divergiert für } |x| > 1
    \end{aligned}\]
    Untersuche Randwerte für $x = \pm 1$
    \[\begin{aligned}
        &\bullet x=  1: &&P( 1)= \sum_{k=1}^\infty \frac{1}{k} \text{ divergent (harmonische Reihe)} \\
        &\bullet x= -1: &&P(-1)= \sum_{k=1}^\infty \frac{(-1)^k}{k} = \sum_{k=0}^\infty \frac{(-1)^{k+1})}{k+1} \\
            &&&= -\underbrace{\qt{\sum_{k=0}^\infty \frac{(-1)^k}{k+1}}}_{\mathclap{\text{konvergent (2.12d)}}}
    \end{aligned}\]
    $\imp P(-1)$ konvergent

    Insgesamt: $P(x)$ konvergent für $[-1, 1)$, divergent für $|x|>1$ und $x=1$.

    \subsection{Satz: Formel von Euler}
    Sei $(a_k)_{k>0}$ Folge in $\IR, a_k \neq 0 \quad \forall k \in \IN_0$, \\
    $\rho$ Konvergenzradius von $P(x) = \sum\limits_{k=0}^\infty a_k (x-x_0)^k$.
    
    Ist $\qt{\abs{\frac{a_k}{a_{k-1}}}}_{k \geq 0}$ konvergent oder bestimmt gegen $+\infty$ \\
    divergent, so ist $\rho = \lim\limits_{k \to \infty}\abs{\frac{a_k}{a_{k+1}}}$

    \textbf{Beweis: } Wende auf $P(x)$ das Quotientenkriterium 2.16 an. $\qed$

    \subsection{Beispiel: Exponentialfunktion}
    \[\begin{aligned}
        &\sum_{k=0}^\infty \frac{x^k}{k!} \text{ konvergent } \forall x \in \IR: \\
        &\abs{\frac{a_k}{a_{k+1}}} = \frac{1}{k\cancel{!}} \cdot \frac{(k+1)\cancel{!}}{1} = k+1 \xrightarrow[k \to \infty]{} \infty \\
        &\underset{3.10}{\imp} \rho = \infty
    \end{aligned}\]
    Man definiert: $\exp: \IR \to \IR$ mit $\displaystyle \exp(x) = \sum_{k=0}^\infty \frac{x^k}{k!}$ (Exponentialreihe)

    Man kann zeigen:
    \begin{enumerate}
        \item $\exp(x+y) = \exp(x) + \exp(y) \quad \forall x,y \in \IR$ (mit Cauchy-Produkt, hier nicht)
        \item $\exp(x) = e^x, e \approx 2,718$ (Eulersche Zahl)
    \end{enumerate}
    Aus 2.: $\displaystyle  \qquad e = \exp(1) = \sum_{k=0}^\infty \frac{1}{k!} = 1 + 1 + \frac{1}{2!} + \frac{1}{3!} + ...$

    \subsection*{Exkurs: Wie erhält man $\exp(x) = e^x$ ?}
    \begin{enumerate}
        \item Definiere: $\quad e := \lim\limits_{n \to \infty} \qt{1 + \frac{1}{n}}^n$ (1.28)
        \item Zeige: $\quad \exp(1) = e = \lim\limits_{n \to \infty} \qt{1 + \frac{1}{n}}^n$ (später)
        \item Zeige, dass Exponentialgesetze für $\exp(x)$ gelten: 
        
        $\exp(x+y) = \exp(x) + \exp(y) \quad \forall x,y \in \IR$ (hier nicht)
        \item Definiere: $\quad e^x = \exp(x) \quad \forall x \in \IR$
    \end{enumerate}
    
    Dies stimmt dann wegen 3. mit den bekannten Rechenregeln für Potenzen und Wurzen überein:
    \begin{itemize}
        \item $e^n = (\exp(1))^n = \exp(n)$
        \item $\qt{\exp\qt{\frac{n}{m}}}^m = exp(n) = e^n \quad \bigm| \sqrt[n]{\ }$
        
        $\imp \exp\qt{\frac{n}{m}} = (e^n)^{\frac{1}{m}} = e^\frac{n}{m} \quad \forall n,m \in \IN$
    \end{itemize}
    Für irrationale Zahlen wird $e^x$ dann mit Hilfe von $e^x = \exp(x)$ berechnet.
        
    So kann auch ein Computer z.B: $e^\pi$ berechnen, indem $\exp(\pi)$ ermittelt wird.

    \subsection{Bemerkung}
    \begin{enumerate}[a)]
        \item Außer der Funktion $e^x$ gibt es auch andere Funktionen die sich als Reihe
        darstellen lassen, z.B wird in Mathe III gezeigt, dass
        $$\begin{aligned}
            &cos(x) = \sum_{n=0}^\infty (-1)^n \frac{x^{2n}}{(2n)!} \\
            &sin(x) = \sum_{n=0}^\infty (-1)^n \frac{x^{2n + 1}}{(2n + 1)!}
        \end{aligned}$$

        \item Wie Beispiel 3.9 zeigt, ist auf dem Rand des Konvergenzintervalls keine
        allgemeine Aussage über das Konvergenzverhalten der entsprechenden Potenzreihe möglich.
        Für $\rho \neq \infty$ müssen die Randwerte gesondert untersucht werden.
    \end{enumerate}
\ifdefined\MAINDOC\else
\end{document}
\fi