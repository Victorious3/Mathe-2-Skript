\ifdefined\MAINDOC\else
\documentclass[10pt,a4paper]{article}
\usepackage{base}

\begin{document}
    \title{Skript Mathe 2}
    \date{18. April 2018}
    \maketitle
\fi
    \subsection{Satz}
    Jede konvergente Folge ist beschränkt.

    \textbf{Beweis:} Sei $(a_n)$ eine konvergente Folge mit Limes $a \in \IR$.

    Zu zeigen:
    $\abs{a_n} \leq K \ \forall a \in \IN$, für ein $K \geq 0$. 
    
    % TODO: Formatting
    Sei $\epsilon = 1$, $(a_n)$ konvergent. 
    $$\imp \abs{a_n} = \abs{a_n - a + a} \leq \underbrace{\abs{a_n - a} + \abs{a}}_{\text{Dreiecksungleichung}}
        < 1 + \abs{a} \ \forall n \geq N$$
    Setze $K = max \{1 + \abs{a}, \abs{a_1}, \abs{a_2}, ..., \abs{a_{N-1}}\}$
    $$\imp \abs{a_n} \leq K \ \forall n \in \IN \qed$$

    \subsection{Bemerkung}
    Wegen 1.8: $(a_n)$ unbeschränkt $\imp (a_n)$ divergent.

    Unbeschränkte Folgen sind also immer divergent.

    \subsection{Beispiel: Geometrische Folge}
    $$\text{Für } q \in \IR: \lim_{n \to \infty} q^n =
    \begin{cases}
        0, \text{falls } \abs{q} < 1 \\
        1, \text{falls } q = 1
    \end{cases}$$
    Für $\abs{q} > 1$ oder $q = -1$ ist $(q^n)$ divergent.

    \textbf{Beweis:}
    \begin{enumerate}[1.)]
        \item $\abs{q} < 1$. Sei $\epsilon > 0$ beliebig.
        Dann ist
        $$\begin{aligned}
            &(q^n - 0) = \abs{q}^n < \epsilon \eqv n \cdot \ln\ \abs{q} < \ln(e) \quad | : \ln(q) < 0 \\
            \eqv &\ n > \frac{\ln(\epsilon)}{\ln\ \abs{q}}
        \end{aligned}$$
        $$\text{Für } N > \frac{\ln(\epsilon)}{\ln\ \abs{q}}: \abs{q}^n < \epsilon \quad \forall n \geq N$$
        \item $q = 1. \ q^n = 1 \quad \forall n \in \IN \imp q^n \to 1$
        \item $\abs{q} > 1 \imp (q^n)$ unbeschränkt $\underset{1.9}{\imp} (q^n)$ divergent
        \item $q = -1 \imp q^n = (-1)^n$. Beweis der Divergenz später (Cauchyfolgen)
    \end{enumerate}

    \subsection{Beispiel}
    Wegen 1.10 sind $(\frac{1}{2^n})_{n \in \IN}$ und $((\frac{-7}{8})^n)_{n \in \IN}$ Nullfolgen.

    \begin{tikzpicture}
        \begin{axis}[
            sequence axis,
            width = 0.5\textwidth,
            samples at = {1, ..., 10},
        ]    
        \addplot[sequence plot, color = blue] {((-7)/8)^x};
        \addplot[sequence plot, color = red] {1/(2^x)};         
        \end{axis}
    \end{tikzpicture}

    \subsection{Bemerkung: Dreiecksungleichung}
    Um Rechenregeln für Folgen in 1.13 beweisen zu können, braucht man folgende Version der
    $\Delta$-Ungleichung:
    $$\begin{aligned}
        &\abs{\abs{a} - \abs{b}} \leq \abs{a-b} \quad \forall a,b \in \IR \text{, da:} \\
        &\bullet\abs{a-b+b} \leq \abs{a-b} + \abs{b} &|\ \abs{-b} \\
        &\eqv \abs{a} - \abs{b} \leq \abs{a-b} \\
        &\bullet\abs{b-a+a} \leq \abs{b-a} + \abs{a} &|\ \abs{-a} \\
        &\eqv \abs{b} - \abs{a} \leq \abs{b-a} \\
        &\boxed{\imp \abs{\abs{a} - \abs{b}} \leq \abs{a-b}}
    \end{aligned}$$

    \subsection{Rechenregeln für Folgen}
    Seien $(a_n), (b_n)$ konvergente Folgen mit $\lim\limits_{n \to \infty} (a_n) = a$ und
    $\lim\limits_{n \to \infty} (b_n) = b$.

    Dann gilt:
    \begin{enumerate}[1.)]
        \item $\lim\limits_{n \to \infty}(a_n + b_n) = a + b$
        \item $\lim\limits_{n \to \infty}(\lambda \cdot a_n) = \lambda \cdot a \quad \forall \lambda \in \IR $
        \item $\lim\limits_{n \to \infty}(a_n \cdot b_n) = a \cdot b$
        \item $$\begin{aligned}
            b \neq 0 \imp & \bullet \: \exists k \in \IN : b_n \neq 0 \: \forall n \geq k \\
                          & \bullet \qt{\frac{a_n}{b_n}}_{n \geq k} \text{ konvergiert gegen } \frac{a}{b}
        \end{aligned}$$
        \item $\lim\limits_{n \to \infty}\abs{a_n} = \abs{a}$
    \end{enumerate}
    Seien weiter $(d_n), (e_n)$ reelle Folgen, $(d_n)$ ist Nullfolge
    \begin{enumerate}[1.), resume]
        \item $(e_n)$ beschränkt $\imp (d_n \cdot e_n)$ ist Nullfolge
        \item $\abs{e_n} \leq d_n \imp \abs{e_n}$ ist Nullfolge
    \end{enumerate}
    \bigbreak
    \textbf{Beweis:}
    \begin{enumerate}[1.)]
        \item $$\begin{aligned}
            &\text{Sei } \epsilon > 0 \imp \exists N_a, N_b \in \IN: \\
            &\bullet \abs{a_n - a} \leq \frac{\epsilon}{2} \quad \forall n \geq N_a \\
            &\bullet \abs{b_n - b} \leq \frac{\epsilon}{2} \quad \forall n \geq N_b \\
            &\imp \abs{a_n + b_n - (a + b)} \leq \underbrace{\abs{a_n - a}}_{\leq \frac{\epsilon}{2}} 
            + \underbrace{\abs{b_n - b}}_{\leq \frac{\epsilon}{2}} < \epsilon \\
            &\forall n \geq \max\{N_a,N_b\}
        \end{aligned}$$

        \item \begin{itemize}
            \item Für $\lambda = 0$ gilt auch $\lambda \cdot a_n \to 0 = \lambda \cdot a$ \checkmark
            \item Für $\lambda \neq 0$: Sei $\epsilon > 0$
            $$\begin{aligned}
                &\imp \exists N \in \IN : \abs{a_n - a} \leq \frac{\epsilon}{\abs{x}} \quad \forall n \geq N \\
                &\imp \abs{\lambda a_n - \lambda a} = \abs{\lambda} \cdot \abs{a_n - a} < \epsilon \quad \forall n > N \checkmark
            \end{aligned}$$
        \end{itemize}
        
        % TODO: Formatting
        \item $$\begin{aligned}
            &\text{Satz 1.8 } \imp (b_n) \text{ beschränkt.} \\
            &\imp \exists k \geq 0: \abs{b_n} \leq k \quad \forall n \in \IN \\
            &\imp \abs{a_n b_n - ab} = \abs{(a_n - a)b_n + a(b_n - b)} \\
            &\leq \abs{a_n-a} \cdot k + \abs{a} \cdot \abs{b_n - b} \quad \text{(*)} \\
            &\text{Sei } \epsilon > 0 \imp &\exists N_a, N_b \in \IN : \abs{a_n - a} < \frac{\epsilon}{2k} \quad \forall n \geq N_a \\
            &   &\abs{b_n - b} < \frac{\epsilon}{2\abs{a}} \quad \forall n \geq N_b \\
            &\underset{(*)}{\imp} \abs{a_n b_n - ab} < \frac{\epsilon}{2k} \cdot k + \abs{a} \cdot \frac{\epsilon}{\abs{a}} = \epsilon \\
            &\forall n \geq \max\{N_a, N_b\}
        \end{aligned}$$

        \item \begin{itemize}
            \item Z.z: $\exists k \in \IN : b_n \neq 0 \quad \forall n \geq k$

            Es ist $b \neq 0$ und $\abs{b} > 0$.

            $$\begin{aligned}
                &\imp \exists l \in \IN : \underbrace{\abs{b_n - b}}_{\underset{1.12}{\geq} \abs{b} - \abs{b_n}} < \frac{\abs{b}}{2} \quad \forall n \geq b \\
                &\imp \exists \abs{b} - \abs{b_n} < \frac{\abs{b}}{2} \quad \forall n \geq k \\
                &\imp \frac{\abs{b}}{2} < \abs{b_n} > 0 \quad \forall n \geq k \ (**) \\
                &\imp b_n \neq 0 \quad \forall n \geq k
            \end{aligned}$$
            \item Z.z: $\qt{\frac{a_n}{b_n}}_{n \geq k}$ hat $\frac{a}{b}$ als Limes.
            
            \newtikzmark
            Da $\frac{a_n}{b_n} = a_n \cdot \frac{1}{b_n}$, genügt es wegen 3.) zu zeigen, dass $\frac{1}{b_n} \to \frac{1}{b}$.
            $$\begin{aligned}
                \text{Sei } \epsilon > 0 &\imp \exists N \in \IN : \underbrace{\abs{b_n - b} < \frac{\epsilon}{2} \cdot \abs{b}^2}_{\tikzmark{a}} \\
                &\imp \abs{\frac{1}{b_n} - \frac{1}{b}} = \abs{\frac{b - b_n}{b \cdot b_n}} \underset{(**)}{<} \frac{2}{\abs{b}^2} \cdot \abs{b - b_n} \tikzmark{b} < \epsilon \quad \forall n \geq N
            \end{aligned}$$
            % Arrow 
            \begin{tikzpicture}[remember picture, overlay]
                \draw[->, line width = 1pt] ([yshift = 7pt]pic cs:a) -- (pic cs:a) -| ([shift = {(7pt, 7pt)}]pic cs:b);
            \end{tikzpicture}
        \end{itemize}
        \item mit 1.12
        \item[6,7.)] Übung
    \end{enumerate}

    \subsection{Beispiele: Rechenregeln}
    \begin{enumerate}[a)]
        \item $$\begin{aligned}
            \frac{(-1)^n + 5}{n} &= ((-1)^n + 5) \cdot \frac{1}{n} \xrightarrow[n \to \infty]{} 0 \text{ wegen 1.13/6} \\
                                 &\bullet \frac{1}{n} \to 0 \\
                                 &\bullet \abs{(-1)^n + 5} \leq \abs{(-1)}^n + 5 = 6 \\
                                 &\imp (-1)^n + 5 \text{ beschränkt}
        \end{aligned}$$
        \item $$\begin{aligned}
            &\frac{3n^2 + 1}{-n^2 + n} \to -3 \text{, denn } \lim_{n \to \infty}
            \frac{3n^2 + 1}{-n^2 + n} = \lim_{n \to \infty} \frac{\cancel{n^2}\qt{3 + \frac{1}{n^2}}}{\cancel{n^2}\qt{-1 + \frac{1}{n}}} \\
            &\underset{1.13/4}{=} \frac{\lim 3 + \frac{1}{n^2}}{\lim -1 + \frac{1}{n}}
            \underset{1.13/1}{=} \frac{3 + \lim \frac{1}{n^2}}{-1 + \lim \frac{1}{n}} = \frac{3}{-1} = -3
        \end{aligned}$$
        \item Sei $x \in \IR$ mit $\abs{x} > 1$ und $k \in \IN_0$.
        
        \newtikzmark
        Dann: $$
            \frac{\tikzmark{a}\boxed{n^k}}{\tikzmark{b}\boxed{X^n}} \xrightarrow[n \to \infty]{} 0
        $$
        \begin{tikzpicture}[remember picture, overlay]
            \draw[->, line width = 1pt] ([shift = {(-15pt, 15pt)}]pic cs:a) node[left = 0]{kte Potenz} -- ([shift = {(-1pt, 5pt)}]pic cs:a);
            \draw[->, line width = 1pt] ([shift = {(-15pt, -5pt)}]pic cs:b) node[left = 0]{exponentielles Wachstum} -- ([shift = {(-1pt, 5pt)}]pic cs:b);
        \end{tikzpicture}
    \end{enumerate} 
\ifdefined\MAINDOC\else
\end{document}
\fi