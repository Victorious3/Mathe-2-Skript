\ifdefined\MAINDOC\else
\documentclass[10pt, a4paper, fleqn]{article}
\usepackage{base}

\begin{document}
    \title{Skript Mathe 2}
    \date{4. Juni 2018}
    \maketitle
\fi

Existenz von $a$ bedeutet: Wenn $x$ nahe genug bei $X_0$ ist, so ist auch $f(x)$ sehr
nahe an $a$.

\textbf{Beweis: } %TODO: Formatting

$(\Leftarrow):$ Gelte (*). Sei $(X_n)$ in $D \setminus \{X_0\}, X_n \to X_0.$
Z.z.: $f(X_n) \to a$

Da $X_n \to X_0$, gibt es $N \in \IN$ mit $|X_n - X_0| < \delta \quad \forall n \geq N$ (1.5) \\
$(*) \imp |f(X_n) - a| < \epsilon \quad \forall n \geq N$ \\
$\underset{1.5}{\imp} f(X_n) \xrightarrow[n \to \infty]{} q$

$(\Rightarrow):$ Mit Kontraposition: Gelte (*) nicht. \\
$\imp \exists \epsilon > 0$ derart, dass für jedes $n \in \IN$ ein $X_n \in D \setminus \{X_0\}$
existiert mit \\
$|X_n - X_0| < \delta$ und $|f(X_n) - a| \geq \epsilon$. \\
$\underset{1.5}{\imp} f(X_n) \ \cancel{\to} \ n$ für $X_n \to X_0. \qed$

\subsection{Beispiel}
$f: \IR \to \IR, f(x) = ax + b$ mit $a, b \in \IR$. Es ist $\lim\limits_{x \to X_0} f(x) = f(X_0).$

Prüfe mit $\epsilon$--$\delta$--Kriterium:

Sei $\epsilon > 0$. Für $\delta = \dfrac{\epsilon}{|a|}$ ist \\
$|f(x) - f(X_0)| = ax + b - aX_0 - b = |a| \cdot \underbrace{|x - X_0|}_{< \delta} < |a| \cdot \dfrac{\epsilon}{|a|} = \epsilon$

\subsection{Definition: Grenzwert II}
Sei $X_0$ HP von $D \subseteq \IR$ und $f: D \to \IR$.
\begin{enumerate}[1.]
    \item $f$ hat in $X_0$ den Grenzwert $+\infty \ (-\infty) :\eqv f(X_n) \to +\infty (-\infty)$ für
    jede Folge $(X_n)$ in $D \setminus \{X_0\}$ mit $X_n \to X_0$.

    Schreibweise: $\lim\limits_{x \to X_0} f(x) = +\infty \ (-\infty)$

    \item Ist $\sup D = \infty \ (\inf D = -\infty)$, so hat $f(x)$ \\
    Limes $a \in \IR$ für $x \to \infty \ (x \to -\infty) :\eqv f(X_n) \to a$ für jede Folge
    in $D$ mit $X_n \to \infty \ (X_n \to -\infty)$
\end{enumerate}

\subsection{Beispiele}
\begin{enumerate}[a)]
    \item $f: \IR \setminus \{0\} \to \IR, f(x) = \dfrac{1}{x^2}$
    \begin{enumerate}[1.]
        \item $\displaystyle \lim_{x \to 0} \frac{1}{x^2} = \infty$, da für jede Nullfolge $(X_n)$ \\
        in $\IR \setminus \{0\}$ gilt: $\dfrac{1}{\underbrace{{X_n}^2}_{=0}} \xrightarrow[n \to 0]{} + \infty$
        \item $\displaystyle \lim_{x \to \infty} \frac{1}{x^2} = 0$, da für jedes $(X_n)$ \\
        in $\IR$ mit $X_n \to \infty: \dfrac{1}{{X_n}^2} \xrightarrow[n \to \infty]{} 0$
    \end{enumerate}

    \item Es gilt für jedes $m \in \IN_0$:
    \begin{enumerate}[1.]
        \item $\displaystyle \lim_{x \to \infty} \frac{\exp(x)}{x^m} = \infty$
        \item $\displaystyle \lim_{x \to -\infty} x \cdot \exp(x) = 0$
    \end{enumerate}
    \textbf{Beweis: }
    \begin{enumerate}[1.]
        \item $\displaystyle
            \exp(x) = \sum_{k = 0}^\infty \frac{x^k}{k!} \geq \frac{X^{m+1}}{(k+1)!} \quad \forall x \geq 0
        $ \\ $\displaystyle 
            \imp \frac{\exp(x)}{x^m} \geq \frac{x^{\cancel{m} + 1}}{(k+1)x^{\cancel{m}}} = \frac{x}{(k+1)!} \to \infty
        $ \\
        für $x \to \infty$
        \item $\displaystyle
            x^m \cdot \exp(x) = \frac{(-1)^m (-x)^m}{\exp(-x)} = (-1)^m \cdot \frac{1}{\boxed{\frac{\exp(-x)}{(-x)^m}} \underset{1.}{\to} \infty}
        $ \\
        für $x \to -\infty$
    \end{enumerate}
\end{enumerate}

\subsection{Definition: Rechts--/Linksseitiger Grenzwert}
\begin{enumerate}[1.]
    \item Ist $X_0$ HP von $D \cap (X_0, \infty)$, so hat $f$ in $X_0$ den rechtsseitigen Grenzwert
    $a \in \IR :\eqv f(X_n) \to a$ für jede Folge $(X_n)$ in $D \cap (X_0, \infty)$ mit $X_n \to X_0$.

    Schreibweise: $\lim\limits_{x \to {X_0}^+} f(x) = a$

    \item Ist $X_0$ HP von $D \cap (-\infty, X_0)$, so hat $f$ in $X_0$ den linksseitigen Grenzwert
    $a \in \IR :\eqv f(X_n) \to a$ für jede Folge $(X_n)$ in $D \cap (-\infty, X_0)$ mit $X_n \to X_0$.

    Schreibweise: $\lim\limits_{x \to {X_0}^-} f(x) = a$
\end{enumerate}

\subsection{Beispiel}
$f: \IR \to \IR \quad x \mapsto \begin{cases}
    -1 & x < 0 \\
    1 & x \geq 0
\end{cases}$

\begin{itemize}
    \item $\lim\limits_{x \to 0^+} f(x) = 1$, da $f(X_n) = 1 \to 1$ \\
    für $(X_n)$ in $(0, \infty)$ und $(X_n) \to 0$

    \item $\lim\limits_{x \to 0^-} f(x) = -1$, da $f(X_n) = -1 \to -1$ \\
    für $(X_n)$ in $(-\infty, 0)$ und $(X_n) \to 0$
\end{itemize}

\subsection{Bemerkung}
Aus 5.11 ist ersichtlich: Der Grenzwert einer Funktion $f$ in $X_0$ existiert $\eqv$ Der Links--
und Rechtsseitige Grenzwert von $f$ in $X_0$ existieren und übereinstimmen.

\subsection{Beispiele}
\begin{enumerate}[a)]
    \item $\lim_{x \to 0} \frac{1}{|x|} = \infty$, aber $\lim_{x \to 0} \frac{1}{x}$ existiert nicht, \\
    da $\lim_{x \to 0^+} \frac{1}{x} = +\infty \neq \lim_{x \to 0^-} \frac{1}{x} = - \infty$

    \item $\lim_{x \to \infty} x = \infty$, $\lim_{x \to -\infty} x = -\infty$
\end{enumerate}

\subsection{Definition: Stetigkeit}
Sei $f: D \to \IR, D \subseteq \IR$
\begin{enumerate}[a)]
    \item f heißt stetig in $X_0 \in D$, falls
    \[
        \underbrace{\lim_{x \to X_0} f(x)}_A \underbrace{= f(X_0)}_B    
    \]
    \item $f$ heißt stetig, falls $f$ in jedem Punkt $X_0 \in D$ stetig ist.
\end{enumerate}

\subsection{Bemerkung}
\begin{enumerate}[a)]
    \item In 5.15a prüft man zwei Bedingungen: A) Der Grenzwert von $f$ in $X_0$ existiert und 
    B) ist gleich $f(X_0)$.

    \item Wegen 5.6 ist $f$ in $X_0 \in D$ stetig $\eqv$
    \[
        \forall \epsilon > 0 \ \exists \delta > 0 \ \forall x \in D: |x-X_0| < \delta
        \imp |f(x) - f(X_0)| < \epsilon    
    \]
\end{enumerate}

\subsection{Beispiele}
\begin{enumerate}[a)] 
    \item $f: \IR \to \IR, f(x) = x^2$ ist in jedem $X_0 \in D$ stetig:

    $\lim\limits_{x \to X_0} f(x) = f(X_0)$, da für $(X_n)$ in $D \setminus \{X_0\}$ gilt:
    \[
        \underbrace{f(X_n) = X_n^2 \to {X_n}^2 \to {X_0}^2}_{A} \underbrace{= f(x)}_{B}    
    \]
    \item Wegen 5.4 ist $f: \IR \to \IR$ mit $f(x) = ax + b$ stetig.
\end{enumerate}

\subsection{Satz}
Sei $f: D \to \IR, D \subseteq \IR$.

Gibt es ein k > 0 mit $|f(x) - f(X_0)| \leq k \cdot |x - X_0| \quad \forall x \in D$, \\
so ist $f$ stetig in $X_0$.

\textbf{Beweis: }
Sei $\epsilon > 0$. Wähle $\delta = \dfrac{\epsilon}{\delta}$
\[
    \imp |f(x) - f(X_0)| \leq k \cdot |\underbrace{x - X_0}_{< \delta}| < k \cdot \delta = \epsilon \qed   
\]

\ifdefined\MAINDOC\else
\end{document}
\fi