\ifdefined\MAINDOC\else
\documentclass[10pt, a4paper, fleqn]{article}
\usepackage{base}

\begin{document}
    \title{Skript Mathe 2}
    \date{25. April 2018}
    \maketitle
\fi
    \textbf{Beweis: }
    \begin{enumerate}
        \item Sei $(a_n)\nearrow$ und nach oben beschränkt

        und seien $a = \sup\{a_n|n \in \IN\}$ und $\epsilon > 0$.

        $\imp a_n \leq a \quad \forall n \in \IN$

        $a$ kleinste obere Schranke

        $\imp a - \epsilon$ keine obere Schranke.

        $\imp \exists N \in \IN : a - \epsilon < a_N \leq a$

        $\underset{\substack{a_n \geq a_N \\ \forall n \geq N}}{\imp} \abs{a_n - a} = a - a_n \leq a - a_N$
        
        $\imp a_n \to a$
        \item analog $\qed$
    \end{enumerate}

    \subsection{Bernoulli-Ungleichung}
    Im folgenden Beispiel wird die Bernoulli-Ungleichung benötigt:
    $$(1 + h)^n \geq 1 + nh \quad \forall h \geq -1 \forall n \in \IN$$
    Beweis mit vollständiger Induktion

    \subsection{Beispiel: Folgen mit Grenzwert $e$}
    % a_n = (1 + 1/n)^n
    % b_n = (1 + 1/n)^(n + 1)
    \begin{tikzpicture}
        \begin{axis}[
            sequence axis,
            width = 0.8\textwidth,
            height = 6cm,
            samples at = {1, ..., 20},
            xtick = {1, 5, 10, 15, 20},
            clip = false
        ]    
        \addplot[color = red] {(1 + 1/x)^x};
        \addplot[color = blue] {(1 + 1/x)^(x + 1)};
        \draw (1, 2.71) -- (20, 2.71) node[xshift = 6pt] {$e$};
        \legend {
            $\qt{1 + \frac{1}{n}}^n$,
            $\qt{1 + \frac{1}{n}}^{(n+1)}$
        }
        \end{axis}
    \end{tikzpicture}
    \begin{itemize}
        \item $a_n = \qt{1 + \frac{1}{n}}^n = \qt{1 + \frac{n+1}{n}}$ ist monoton.
        
        Zeigen dazu: $a_n \geq a_{n-1} \qt{\eqv \frac{a_n}{a_{n-1}} \geq 1}$
        $$\begin{aligned}
            \frac{a_n}{a_{n-1}} &= \qt{\frac{n+1}{n}}^n \cdot \qt{\frac{n-1}{n}}^{n-1} \\
            &= \qt{\frac{n+1}{n}}^n \cdot \qt{\frac{n-1}{n}}^n \cdot \frac{n}{n-1}
            = \qt{\frac{n^2-1}{n^2}}^n \cdot \frac{n}{n-1} \\
            &= \qt{1 - \frac{1}{n^2}}^n \qt{\frac{n}{n-1}} \underset{1.24}{\geq} \underbrace{\qt{1 - \frac{1}{n}}}_{\frac{n - 1}{n}} \cdot \frac{n}{n-1} = 1 \\
            h = \frac{1}{n^2}
        \end{aligned}$$

        \item $b_n = \qt{1 + \frac{1}{n}}^{1+n} = \qt{\frac{n+1}{n}_{n+1}}$ ist monoton fallend.

        $$\begin{aligned}
            \text{Zeige dazu: }
            &b_n \leq b_{n-1} \qt{\eqv \frac{b_n}{b_{n-1}} \leq 1 } \\
            \text{Analog: }
            &\frac{b_n}{b_{n-1}} = \qt{1 + \frac{1}{n^2-1}}^n \qt{\frac{n}{n+1}} \\
            \text{Wegen } &\qt{1 + \frac{1}{n^2-1}}^n \underset{1.24}{\geq} 1 + \frac{n}{n^2-1} 
            \geq \underbrace{1 + \frac{1}{n}}_{\frac{n+1}{n}} \text{ ist } \\
            &\frac{b_n}{b_{n-1}} \geq \frac{1+1}{n} \cdot \frac{n}{n+1} = 1 \quad (?)
        \end{aligned}$$
    \end{itemize}

    In Beispiel 1.27 werden wir sehen, dass
    $$\lim_{n \to \infty} a_n = \lim_{n \to \infty} b_n$$
    Der Limes wird als Eulerische Zahl $e$ bezeichnet. Dazu zunächst:

    \subsection{Satz: Intervallschachtelung}
    Seien $(a_n),  (b_n)$ reelle Folgen mit
    \begin{itemize}
        \item $(a_n)\nearrow$, $(b_n)\searrow$
        \item $a_n \leq b_n \quad \forall n \in \IN$
        \item $b_n - a_n \to 0$
    \end{itemize}
    Dann sind $(a_n), (b_n)$ konvergent und besitzen den selben Limes.

    \textbf{Beweis: } Es ist $a_1 \leq a_n \leq b_n \leq b_1 \quad \forall n \in \IN$
    
    \begin{tabular}{rl}
        $\imp$ & $(a_n)$ hat obere Schranke $b_1$ \\
        & $(b_n)$ hat untere Schranke $a_1$ \\
        $\underset{1.23}{\imp}$ & $(a_n), (b_n)$ konvergent.
    \end{tabular}

    Da $(b_n - a_n)$ Nullfolge, sind auch die Grenzwerte gleich. $\qed$

    \subsection{Beispiel}
    \begin{itemize}
        \item $(a_n)\nearrow, (b_n)\searrow$ (siehe 1.25)
        \item $\underline{(a_n)} = \qt{1 + \frac{1}{n}}^n \underline{\leq} \qt{1 + \frac{1}{n}} \cdot a_n = \qt{1 + \frac{1}{n}}^{n+1} = \underline{b_n}$
        \item $\lim\limits_{n \to \infty} b_n = \lim\limits_{n \to \infty} \underbrace{\qt{1 + \frac{1}{n}}}_{\to 1} \cdot a_n \underset{1.13/3}{=} \lim\limits_{n \to \infty} a_n$
    \end{itemize}

    \subsection{Definition: Eulersche Zahl}
    $$
        e := \lim_{n \to \infty} \qt{1 + \frac{1}{n}}^n \qt{= \lim_{n \to \infty} \qt{1 + \frac{1}{n}}^{n+1} }
    $$

    \subsection{Bemerkung}
    $(a_n)$ konvergent $\underset{1.8}{\imp} (a_n)$ beschränkt.
    \textbf{Die Umkehrung gilt nicht!} \\
    z.B besitzt jedoch $a_n = (-1)^n$ zwei konvergente Teilfolgen mit Limes $+1$ und $-1$.

    \subsection{Definition: Teilfolge}
    Sei $(a_n)_{n \in \IN}$ eine Folge und $(n_k)_{k \in \IN}$ eine streng monoton steigende Folge
    von Indizes. Dann heißt die Folge $(a_{n_k})_{k \in \IN}$ Teilfolge von $(a_n)_{n \in \IN}$.

    \subsection{Beispiel}
    $a_n = (-1)^n$
    \begin{itemize}
        \item $n_k = 2k \imp a_{n_k} = a_{2k} = (-1)^{2k} = 1 \quad \forall k \in \IN$
        \item $n_k = 2k + 1 \imp a_{n_k} = a_{2k + 1} = (-1)^{2k + 1} = -1 \quad \forall k \in \IN$
    \end{itemize}

    \subsection{Bemerkung}
    $(a_n)$ konvergiert gegen $a$
    $\imp$ Jede Teilfolge von $(a_n)$ konvergiert gegen $a$.

    \subsection{Definition: Häufungspunkt (HP)}
    Sei $(a_n)$ reelle Folge. $h \in \IR$ heißt Häufungspunkt von $(a_n)$, wenn es eine
    Teilfolge von $(a_n)$ gibt, die gegen $h$ konvergiert.

    \subsection{Beispiel}
    $(a_n)$ mit $a_n = (-1)^n + \frac{1}{n}$ hat zwei Häufungspunkte: $-1$ und $1$.

    \subsection{Satz: Bonzano-Weierstraß}
    Sei $(a_n)$ reelle Folge.
    $(a_n)$ beschränkt $\imp (a_n)$ besitzt konvergente Teilfolge

    \textbf{Beweis: } Konstruiere konvergente Teilfolge $({a_n}_k)_{k \in \IN}$,

    $(a_n)$ beschränkt $\imp \abs{a_n} \leq K \quad \forall n \in \IN$ (K geeignet)
    
    $\imp a_n \in \underbrace{[-K, K]}_{= [A_0, B_0]} \quad \forall n \in \IN$
    \begin{itemize}
        \item \underline{$k = 1$}: Halbiere $[A_0, B_0]$
        \begin{itemize}
            \item Falls in der linken Folgenhälfte unendlich viele Folgeglieder liegen, wähle eines davon aus.
            \item Falls nicht, liegen in der rechten Hälfte unendlich viele. Wähle eines davon aus.
        \end{itemize}
        \item[] Das ausgewählte Folgenglied nennen wir ${a_n}_1$, die Intervallhälfte aus der es stammt
        $[A_1, B_1]$.
        \item \underline{$k = 2$}: Halbiere $[A_1, B_1]$. Wende obiges Verfahren an, um
        ${a_n}_2 \in [A_2, B_2]$ zu bestimmen.
        \item usw ...
    \end{itemize}
    Erhalte Intervallschachtelung mit
    \begin{itemize}
        \item $(A_k)\nearrow, (B_k)\searrow$
        \item $A_k \leq B_k$
        \item $A_k = B_k = \frac{K}{2^{k-1}} \to 0$
    \end{itemize}
    $\underset{1.26}{\imp} \lim\limits_{k \to \infty} A_k = \lim\limits_{k \to \infty} B_k$

    Da $A_k \leq {a_n}_k \leq B_k$, ist $\lim\limits_{n \to \infty} A_k \underset{1.15}{=} \lim\limits_{k \to \infty} (a_{n_k}) \qed$

\ifdefined\MAINDOC\else
\end{document}
\fi