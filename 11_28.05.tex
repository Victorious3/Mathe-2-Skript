\ifdefined\MAINDOC\else
\documentclass[10pt, a4paper, fleqn]{article}
\usepackage{base}

\begin{document}
    \title{Skript Mathe 2}
    \date{28. Mai 2018}
    \maketitle
\fi
    \begin{enumerate}[a), start = 6]
        \item[]
        \begin{itemize}
            \item $n$ gerade
            $\imp f: \IR_{\geq 0} \to \IR_{\geq 0}, x \mapsto x^n$ bijektiv

            In diesem Fall hat die Umkehrfunktion die Vorschrift
            \[
                \sqrt[n]{\ }: \IR_{\geq 0} \to \IR_{\geq 0}, x \mapsto \underbrace{\sqrt[n]{x}}_{\geq 0}
            \]
            \textbf{Achtung: } Wenn n gerade, dann hat $x^n = a$ für gegebenes $a \in \IR$
            \begin{itemize}
                \item keine Lösung, falls $a<0$
                \item genaue eine Lösung, falls $a=0$ und zwar $x=0$
                \item genau zwei Lösungen, falls $a>0$ und zwar

                $x_1 = \underbrace{\sqrt[n]{a}}_{> 0} \quad x_2 = \underbrace{-\sqrt[n]{a}}_{< 0}$
            \end{itemize}
        \end{itemize}
        \item Polynome: $p: \IR \to \IR, x \mapsto a_n x^n + a_{n-1} x^{n-1} + ... + a_0 x^0
            = \sum_{k=0}^n a_k x^k$

        $a_0, ..., a_n \in \IR$ heißen Koeffizienten

        Falls $a_n \neq 0$, so heißt $n$ Grad von $p$, man schreibt $\grad(p) = n$

        Für ein Polynom $p$ von Grad $n$ kann man zeigen:
        \begin{enumerate}[1.]
            \item $p$ besitzt höchstens $n$ Nullstellen
            \item Falls $n$ ungerade, ist $p$ surjektiv und besitzt mindestens eine Nullstelle
            \item Falls $n$ gerade, ist $p$ nicht surjektiv und kann daher auch keine Nullstelle haben
        \end{enumerate}
        Bekannte Verfahren zur Berechnung von Nullstellen:
        \begin{itemize}
            \item $\grad(p) = 2$: Mitternachtsformel/pq-Formel
            \item $\grad(p) \geq 3$: Polynomdivision (Mathe III), numerische Verfahren (z.B Newton-Verfahren)
        \end{itemize}
        \item Rationale Funktionen:

        Quotienten von Polyonmen $p,q$ mit $f: D \to \IR$
        \[
            x \mapsto \frac{p(x)}{q(x)} \qquad D=\{x \in \IR \ | \ q(x) \neq 0\}
        \]

        \item Logarithmen und Exponentialfunktion:

        \begin{enumerate}[1.]
            \item der natürliche Logarithmus:

            Man kann zeigen, dass für die Exponentialreihe unter 3.11 gilt:
            \begin{itemize}
                \item $\exp(\IR) = \IR_{> 0}$
                \item $\exp: \IR \to \IR_{> 0}$ ist bijektiv
            \end{itemize}
            Die Umkehrfunktion von $\exp(x)$ ist der natürliche Logarithmus:
            \[
                \ln: \IR_{> 0} \to \IR, x \mapsto \ln(x)
            \]

            \item Exponentialfunktion:

            Sei $q > 0, q \neq 0$. Für $x \in \IQ, x = \frac{a}{b}$ ist
            $q^x = \sqrt[b]{q^a} \quad a \in \IZ, b \in \IN$

            Mit Hilfe der Funktion $\exp(x), \ln(x)$ kann man Exponentialfunktionen
            zu einer beliebigen gegebenen Basis $q$ und $x \in \IR$ definieren:
            \[
                f: \IR \to \IR_{> 0} \quad x \mapsto q^x := \exp(x \cdot \ln(q))     
            \]

            \item Aus 2. ergibt sich die Regel:
            \[
                \ln(q^x) = x \cdot \ln(q) \quad \forall x \in \IR
            \]

            \item Man kann wegen 2. eine Basis $q$ durch eine beliebige andere Basis ausdrücken,
            z.B: $q^x = e^{x \cdot \ln(q)}$ (da $\exp(x) = e^x$ (3.11))

            \item Logarithmus zur Basis $q > 0, q \neq 1$: Bilde die Umkehrfunktion
            von $f(x) = q^x$ (unter 2.)
            \[
                \log_q: \IR_{>0} \to \IR \quad x \mapsto \log_q(x)    
            \]

            \item $log_q$ lässt sich analog zu 4. durch jeden anderen Logarithmus ausdrücken,
            z.B ist 
            \[
                \ln(x) = \ln(q^{\log_q(x)}) \underset{3.}{=} \log_q(x)
                \eqv \log_q(x) = \frac{\ln(x)}{\ln(y)}
            \]

            \item Rechenregeln:
            \begin{itemize}[$-$]
                \item für $f(x) = q^x$ ergeben sich aus 2. und den Regeln für $\exp(x)$ (3.11):
                \begin{itemize}[$\bullet$]
                    \item $q^{x+y} = q^x \cdot q^y \quad \forall x,y \in \IR$
                    \item $q^{-x} = \dfrac{1}{q^x}$, da $1 = q^{x-x} - q^x \cdot q ^{-x} \quad \forall x \in \IR$
                    \item $(q^x)^y = q^{x \cdot y}$
                    \item $(pq)^x = p^x \cdot q^x$
                \end{itemize}
                \item für $\log_q(x)$ ergeben sich aus denen für $q^x$:
                \begin{itemize}[$\bullet$]
                    \item $\log_q(xy) = \log_q(x) + \log_q(y) \quad \forall x,y > 0$
                    
                    denn für $x = q^u, y = q^v$ ist 
                    
                    $\log_q(xy) = \log_q(q^{u+v}) = u + v = \log_q(x) + \log_q(y)$

                    \item $\log_q\qt{\dfrac{q}{x}} = -\log_q(x) \quad \forall x > 0$

                    [mit $q^v = \log_q(x^\alpha) \underset{3./ 6.}{=} \alpha \cdot \log_q(x) \quad \forall x > 0, \alpha \in \IR$]
                \end{itemize}
            \end{itemize}
        \end{enumerate}
        \item Trigonometrische Funktionen:

        \begin{tikzpicture}
            \begin{axis}[
                clip = false,
                width = 0.7\textwidth,
                height = 0.7\textwidth,
                xmin = -1.2, xmax = 1.2,
                ymin = -1.2, ymax = 1.2,
                xtick distance = 1,
                ytick distance = 1,
                axis x line = center,
                axis y line = center,
                axis line style = {->}
            ]
            \coordinate (A) at (0, 0);
            \coordinate (B) at ({sin(45)}, {sin(45)});
            \coordinate (C) at ({cos(45)}, 0);

            \addplot [domain = 0:360, samples = 100] ({cos(x)}, {sin(x))});
            \draw (A) node[dot]{} 
                -- (B) node[dot]{} 
                    node[right, yshift = 5pt] {$P_\varphi = (\cos(\varphi), \sin(\varphi))$}
                -- (C) node[dot]{}
                -- (A);
            \draw pic["$\varphi$", angle radius = 20pt, draw = black] {angle = C--A--B};
            \draw pic["$\cdot$", draw = black] {angle = B--C--A};
            
            % Braces
            \draw[decorate, decoration = {brace, mirror, amplitude = 10pt, raise = 4pt}] 
                (A) -- (C) node[midway, yshift = -20pt] {$\cos(\varphi)$};
            \draw[decorate, decoration = {brace, mirror, amplitude = 10pt, raise = 4pt}]
                (C) -- (B) node[midway, right, xshift = 12pt] {$\sin(\varphi)$};
            
            \end{axis}
        \end{tikzpicture}

        \begin{tabular}{rl}
            $\varphi$: & Winkel zwischen x-Achse und Strecke $\overline{0 \ P_\varphi}$ \\
            $\cos \varphi$: & Ankathete an $\varphi$ in $\Delta(0 \ A_\varphi \ P_\varphi)$ \\
            $\sin \varphi$: & Gegenkathete an  $\varphi$ in $\Delta(0 \ A_\varphi \ P_\varphi)$
        \end{tabular}

        Daraus ergeben sich die Winkelfunktionen:

        \begin{tabular}{rl}
            $\cos:$ & $\IR \to [-1, 1], x \mapsto \cos(x)$ \\
            $\sin:$ & $\IR \to [-1, 1], x \mapsto \sin(x)$ \\
            $\tan:$ & $\IR \setminus \{(k + \frac{1}{2}) \pi \ | \ k \in \IZ\} \to \IR, x \mapsto \dfrac{\sin(x)}{\cos(x)}$ \\
            $\cotan:$ & $\IR \setminus \{k \pi \ | \ k \in \IZ\} \to \IR, x \mapsto \dfrac{\cos(x)}{\sin(x)}$
        \end{tabular}
    \end{enumerate} % Enumerumeritemitemize
\ifdefined\MAINDOC\else
\end{document}
\fi