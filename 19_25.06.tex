\ifdefined\MAINDOC\else
\documentclass[10pt, a4paper, fleqn]{article}
\usepackage{base}

\begin{document}
    \title{Skript Mathe 2}
    \date{25. Juni 2018}
    \maketitle
\fi

\begin{enumerate}
    \item[]
    \begin{enumerate}
        \addtolength{\itemindent}{12pt}
        \item[\underline{1. Fall}:] Beide Extrema werden auf dem Rand angenommen: \\
        $f(a) = f(b) \imp m = M$ \\
        $\imp f$ konstant $\imp f'(\xi) = 0 \quad \forall \xi \in (a, b)$
        \item[\underline{2. Fall:}] Ein Extremum wird auf dem Rand angenommen: \\
        $\imp \exists \xi \in (a, b): f(\xi)$ Extremum $\underset{6.18}{\imp}
        f'(\xi) = 0$.
    \end{enumerate}
    \item[3.]
    Es ist $g(b) \neq g(a)$, denn sonst gäbe es ein $x \in (a, b)$ mit $g'(x) = 0$ (Rolle)

    Hilfsfunktion: $h(x) = f(x) = \dfrac{f(b) - f(a)}{g(b) - g(a)} \cdot g(x)$

    Es ist $h(b) - h(a) = 0$. $h$ stetig auf $[a, b]$ und differenzierbar in $(a, b)$.
    \[\begin{aligned}
        &\underset{\text{Rolle}}{\imp} \exists \xi \in (a, b): h'(\xi) = 0 \\
        &\imp \frac{f'(\xi)}{g'(\xi)} = \frac{f(b) - f(a)}{g(b) - g(a)}
    \end{aligned}\]
    \item[1.] Folgt aus 3. für $g(x) = x$.
\end{enumerate}
\subsection{Monotoniekriterium}
Sei $f: [a, b]$ stetig und auf $(a, b)$ differenzierbar.
\begin{enumerate}[1.]
    \item $f'(x) \underset{\hbox{$(\leq)$}}{\geq} 0 \quad \forall x \in (a, b)
    \eqv f$ monoton $\underset{\hbox{(fallend)}}{\text{wachsend}}$ auf $[a, b]$
    \item $f'(x) \underset{\hbox{$(<)$}}{>} 0 \quad \forall x \in (a, b)
    \underset{\hbox{$\cancel{\Leftarrow}$}}{\imp} f$ streng monoton $\underset{\hbox{(fallend)}}{\text{wachsend}}$ auf $[a, b]$
    \item $f'(x) = 0 \quad \forall x \in (a, b) \eqv f$ konstant auf $[a, b]$
\end{enumerate}
\textbf{Beweis: }
\begin{enumerate}[1.]
    \item $(\Rightarrow):$ Sei $a \leq x_1 < x_2 \leq b$

    $\underset{6.20.1}{\imp} \exists \xi \in (x_1, x_2): f(x_2) - f(x_1) =
    \underbrace{f'(\xi)}_{\geq 0} \cdot \underbrace{(x_2 - x_1)}_{> 0} \geq 0$

    $\imp f(x_1) \leq f(x_2)$

    $(\Leftarrow):$ Sei $f$ monoton wachsend auf $[a, b]$ und differenzierbar in
    $(a, b)$
    \[
        \imp f'(x) = \lim_{h \to 0} \frac{f(x + h) - f(x)}{h}    
    \]\[\begin{aligned}
        &\text{Da} \ \frac{(f(x + h) - f(x)) \geq 0}{h > 0} \geq 0 \text{ für } h < 0 \text{ und } \\
        &\frac{(f(x + h) - f(x)) \leq 0}{h < 0} \leq 0 \text{ ist }
        f'(x) \geq 0 \quad \forall x \in (a, b)
    \end{aligned}\]

    \item[2. + 3.] analog $\qed$
\end{enumerate}
Bemerkung zu 2.: $f(x) = x^3$ ist streng monoton wachsend aber $f'(0) = 0$

\subsection{Satz: Hinreichende Bedingung für lokale Exterma I}

Sei $f: I \to \IR$ differenzierbar und $x_0 \in I, f'(x_0) = 0$
\bigskip
\begin{enumerate}[1.]
    \begin{minipage}{0.5\textwidth}
        \item \raisebox{-\height + \ht\strutbox}{
            \begin{tikzpicture}
                \begin{axis}[
                    width = 1.1\textwidth,
                    unit vector ratio* = 1 1 1, 
                    axis lines = left,
                    ymin = -4, ymax = 1,
                    xmin = -3, xmax = 3,
                    xtick = {-2, 0, 2},
                    ytick = \empty,
                    xticklabels = {$x_0 - \delta$, $x_0$, $x_0 + \delta$},
                    domain = -3:3
                ]
                \addplot[color = black, samples = 50] {-x^2};
                \draw[dashed] (0, -4) -- (0, 0);
                \draw [decorate, decoration = {brace, amplitude = 10pt, raise = 2pt}, yshift = 0pt]
                    (-1.9, -4) -- (-0.1, -4) node [midway, yshift = 20pt] {$f' \geq 0$};
                \draw [decorate, decoration = {brace, amplitude = 10pt, raise = 2pt}, yshift = 0pt]
                    (0.1, -4) -- (1.9, -4) node [midway, yshift = 20pt] {$f' \leq 0$};
                \end{axis}
            \end{tikzpicture}
        }
    \end{minipage}
    \begin{minipage}{0.5\textwidth}
        \item \raisebox{-\height + \ht\strutbox}{
            \begin{tikzpicture}
                \begin{axis}[
                    width = 1.1\textwidth,
                    axis lines = left,
                    ymin = -2, ymax = 2,
                    xmin = -2, xmax = 2,
                    xtick = {-1.25992, 0, 1.25992},
                    ytick = \empty,
                    xticklabels = {$x_0 - \delta$, $x_0$, $x_0 + \delta$},
                    domain = -2:2
                ]
                \addplot[color = black, samples = 50] {x^3};
                \draw[dashed] (0, -4) -- (0, 0);
                \draw [decorate, decoration = {brace, amplitude = 10pt, raise = 2pt}, yshift = 0pt]
                    ({-2^(1/3) + 0.1}, -2) -- (-0.1, -2) node [midway, yshift = 20pt] {$f' > 0$};
                \draw [decorate, decoration = {brace, amplitude = 10pt, raise = 2pt}, yshift = 0pt]
                    (0.1, -2) -- ({2^(1/3) - 0.1}, -2) node [midway, yshift = 20pt] {$f' > 0$};
                \end{axis}
            \end{tikzpicture}
        }
    \end{minipage}
\end{enumerate}
\bigskip

\begin{enumerate}[1.]
    \item
    $f'(y) \underset{\hbox{$(\leq)$}}{\geq} 0 \quad \forall (x_0 - \delta, x_0)$ und \\
    $f'(y) \underset{\hbox{$(\leq)$}}{\geq} 0 \quad \forall (x_0, x_0 + \delta)$ für ein $\delta < 0$

    $\imp f$ hat ein lokales Minimum (Maximum) in $x_0$.

    \item
    $f'(x) < 0 \quad \forall x \in (x_0 - \delta, x_0) \cup (x_0, x_0 + \delta)$

    [1. hat einen Vorzeichenwechsel, 2. nicht]
\end{enumerate}

\textbf{Beweis: } Für lokales Minimum in $x_0$:

Z.z: $f(x) \geq f(x_0) \quad \forall x \in U := (x_0 - \delta , x_0 + \delta)$

Da $x \in U \setminus {x_0} \underset{6.20.1}{\imp} \exists \xi$ zwischen $x$ und $x_0$; \\
$\xi \neq x_0$, so dass $f(x) - f(x_0) = f'(\xi) \cdot (x - x_0)$ (*)

\begin{enumerate}
    \addtolength{\itemindent}{12pt}
    \item[\underline{1. Fall}: ] $x \in (x_0 - \delta, x_0)$
    \[\begin{aligned}
        &\imp x - x_0 < 0, f'(\xi) \leq 0 \\
        &\underset{\text{(*)}}{\imp} f(x) - f(x_0) \geq 0
        \imp f(x) \geq f(x_0)    
    \end{aligned}\]
    \item[\underline{2. Fall}: ] $x \in (x_0, x_0 + \delta)$
    \[\begin{aligned}
        &\imp x - x_0 > 0, f'(\xi) \geq 0 \\
        &\underset{\text{(*)}}{\imp} f(x) - f(x_0) \geq 0
        \imp f(x) \geq f(x_0)    
    \end{aligned}\]
\end{enumerate}
Insgesamt: $f(x) \geq f(x_0) \quad \forall x \in U$

(Rest analog) $\quad \qed$

\subsection{Bemerkung}
\hfill
\begin{minipage}{0.4\textwidth}
    \begin{tikzpicture}
        \begin{axis}[
            width = 1.2\textwidth,
            axis y line = left,
            axis x line = center,
            ymin = -1.5, ymax = 1.5,
            xmin = -1, xmax = 1,
            xtick = {0},
            xticklabels = {$x_0$},
            ytick = \empty
        ]
        \addplot[color = black, samples = 2] {1.7*x};
        \draw[color = black] plot [smooth, tension = 1.2] coordinates {
            (-2, 0) (-1, -1) (1, 1) (2, 0)
        };
        \end{axis}
        \draw (4.5, 2.9) node {$f'$};
    \end{tikzpicture}    
\end{minipage}
\hfill
\begin{minipage}{0.4\textwidth}
    Vorzeichenwechsel von - nach + \\
    $\imp f$ hat Minimum in $x_0$
\end{minipage}
\bigskip

$f'$ weist in $x_0$ einen Vorzeichenwechsel auf, wenn die Steigung von $f'$ in $x_0$
positiv (negativ) ist, d.h. wenn $f''(x_0) > 0 \ (f''(x_0) < 0)$.

Wenn $f''(x) = 0$, ist über einen Vorzeichenwechsel keine Aussage möglich.
\bigskip

\hfill
\begin{minipage}{0.4\textwidth}
    \begin{tikzpicture}
        \begin{axis}[
            width = 1.2\textwidth,
            axis y line = left,
            axis x line = center,
            xtick = {0},
            xticklabels = {$x_0$},
            ytick = \empty
        ]
        \addplot[color = black] {4 * x^3};
        \addplot[dashed] {x^4};
        \end{axis}
        \draw (1, 1) node {$f'$};
    \end{tikzpicture}
\end{minipage}
\hfill
\begin{minipage}{0.4\textwidth}
    $f''(x_0) = 0$ und VZW
    \[\begin{aligned}
        f(x) &= x^4 \\
        f'(0) &= 0 \\
        f''(0) &= 0
    \end{aligned}\]
\end{minipage}
\bigskip

\hfill
\begin{minipage}{0.4\textwidth}
    \begin{tikzpicture}
        \begin{axis}[
            width = 1.2\textwidth,
            axis y line = left,
            axis x line = center,
            xtick = {0},
            xticklabels = {$x_0$},
            ytick = \empty
        ]
        \addplot[color = black] {3 * x^2};
        \addplot[dashed] {x^3};
        \end{axis}
        \draw (1, 2.4) node {$g'$};
    \end{tikzpicture}
\end{minipage}
\hfill
\begin{minipage}{0.4\textwidth}
    $g''(x_0) = 0$ und kein VZW
    \[\begin{aligned}
        g(x) &= x^3 \\
        g'(0) &= 0 \\
        g''(0) &= 0
    \end{aligned}\]
\end{minipage}

\subsection{Satz: Hinreichende Bedingung für Extrema II}
Sei $f: I \to \IR$ differenzierbar und $x_0 \in I$ 2-mal differenzierbar.

$(f'(x_0) = 0, f''(x_0) \underset{\hbox{$(<)$}}{>} 0) \imp 
f $ hat in $x_0$ ein lokales Minimum (Maximum)

\textbf{Beweis: } Für Minimum:

Es ist $\lim\limits_{h \to 0} \dfrac{f'(x_0 + h) - f(x_0)}{h} =
\lim\limits_{h \to 0} \dfrac{f(x_0 + h)}{h} = f''(x_0) > 0$

$\imp \exists \delta < 0: \dfrac{f'(x_0 + h)}{h} > 0 \quad \forall \ |h| < \delta, h \neq 0 \quad$ (*)
\[\begin{rcases}
    \text{1. Fall}: & -\delta < h < 0 \underset{\text{(*)}}{\imp} f'(x_0 + h) < 0 \\
    \text{2. Fall}: & 0 < h < \delta \underset{\text{(*)}}{\imp} f'(x_0 + h) > 0
\end{rcases} \text{Vorzeichenwechsel}\]
$f'(x_0) = 0$ und Vorzeichenwechsel $\underset{6.22}{\imp} f$ hat ein lokales Minimum in $x_0$.

Rest analog $\qed$

\subsection*{Die Regeln von L'Hospital (1661--1704)}
Problem: Grenzwerte vom Typ $\frac{0}{0}, \ \frac{\infty}{\infty}, \ 0 \cdot \infty, \ 0^0$ usw...

Beispiel: $\dfrac{\sin(x)}{x} \xrightarrow[x \to 0]{}$ ?

\begin{tikzpicture}
    \begin{axis}[
        unit vector ratio* = 1 1 1, 
        axis lines = center,
        ymin = -1, ymax = 3,
        xtick = \empty,
        ytick = \empty,
        domain = -3:7
    ]
    \addplot[color = black, samples = 50] {sin(deg(x))};
    \addplot[color = black] {x};
    \end{axis}
    \draw (7.4, 1.2) node {$\sin(x)$};
    \draw (4.3, 2.9) node {$x$};
\end{tikzpicture}

$f(x) = \sin(x)$ und $g(x) = x$ haben in $x = 0$ \\
die selbe Tangente $(t(x) = x) \imp f, g$ konvergieren \\
mit der gleichen Geschwindigkeit gegen 0, wenn $x \to 0$.

$\imp \dfrac{\sin(x)}{x} \to 1$ für $x \to 0$.

\ifdefined\MAINDOC\else
\end{document}
\fi