\ifdefined\MAINDOC\else
\documentclass[10pt,a4paper]{article}
\usepackage{base}

\begin{document}
    \title{Skript Mathe 2}
    \date{16. April 2018}
    \maketitle
\fi

    \section{Folgen}
    \subsection{Definition}
    Eine Folge $(a_n)_{n \in \IN}$ ist eine Abbildung von den natürlichen Zahlen
    $(\IN)$ in eine beliebige Menge $M$ (oft $M \subseteq \IR$).

    \begin{tabular}{rl}
        $a_n$:& n-tes Folgenglied \\
        $n$:& Index
    \end{tabular}

    Oft ist das erste Folgenglied nicht $a_1$, sondern z.B: $a_7$.
    
    \textbf{Schreibweise:} $(a_n)_{n \in \IN}$, $(a_n)_{n \geq n_0}$ oder $(a_n)$

    \subsection{Beispiele}
    \begin{enumerate}[a)]
        \item $a_n = c\ \forall n \in \IN$ (konstante Folge)
        \item $a_n = n$ (Ursprungsgerade)
        
        \begin{tikzpicture}
            \begin{axis}[
                sequence axis,
                width = 0.5\textwidth,
                xmax = 3.5, ymax = 3.5,
                samples at = {0, ..., 3}
            ]    
            \addplot[sequence plot] {x};      
            \end{axis}
        \end{tikzpicture}

        \item $a_n = (-1)^n, n \in \IN$ (alternierend)

        \begin{tikzpicture}
            \begin{axis}[
                sequence axis,
                width = 0.5\textwidth,
                ymin = -1.5, ymax = 1.5,
                samples at = {0, ..., 5},
                axis x line = center
            ]    
            \addplot[sequence plot] {(-1)^x};         
            \end{axis}
        \end{tikzpicture}

        \item $a_n = \frac{1}{n}$ (Nullfolge)

        \begin{tikzpicture}
            \begin{axis}[
                sequence axis,
                width = 0.5\textwidth,
                ymax = 1.1, ymin = 0,
                samples at = {1, ..., 10},
                minor y tick num = 5
            ]    
            \addplot[sequence plot] {1/x};         
            \end{axis}
        \end{tikzpicture}

        \item \underline{Rekursive Folgen}, z.B: Fiboacci-Folge.

        $f_1 = 1, f_2 = 1, \underbrace{f_{n+1} = f_n + f_{n-1}}_{\text{Rekursionsformel}}$

        $f_3 = 1+1 = 2, f_4 = 3, f_5 = 5, ...$

        \begin{tikzpicture}
            \begin{axis}[
                sequence axis,
                width = 0.5\textwidth,
                ylabel = {$f_n$},
                minor y tick num = 5,
                ymin = 0
            ]    
            \addplot[sequence plot] coordinates {
                (1, 1) (2, 1) (3, 2) (4, 3) (5, 5) (6, 8) (7, 13) (8, 21)
            };         
            \end{axis}
        \end{tikzpicture}

        \item \underline{Exponentielles Wachstum} (z.B von Bakterienstämmen)
        \begin{tabular}{rl}
            $q$:  & Wachstumsfaktor \\
            $X_0$:& Startpopulation
        \end{tabular}

        \textbf{Explizit:} $X_n = q^n * X_0$

        z.B: $X_0 = 5, q = 2$

        $\rightarrow X_1 = 10, X_2 = 20, X_3 = 40, ...$

        \begin{tikzpicture}
            \begin{axis}[
                sequence axis,
                width = 0.5\textwidth,
                samples at = {0, ..., 5}
            ]    
            \addplot[sequence plot] {2^x * 5};         
            \end{axis}
        \end{tikzpicture}

        \item \underline{Logistisches Wachstum}
        
        $X_{n+1} = r \cdot X_n \cdot (1 - X_n)$

        \begin{tabular}{rl}
            $r \in [0, 4]$:   & Wachstums-/Sterbefaktor \\
            $X_n \in [0, 1]$: & Relative Anzahl der Individuen in Generation $n$
        \end{tabular}

        Anzahl der Individuen in Generation $n+1$ hängt ab von der aktuellen Populationsgröße
        $X_n$ und den vorhandenen natürlichen Ressourcen, charakterisiert durch $(1 - X_n)$

        %TODO
        %Bsp: $r = 2, X_n = 0.8$
    \end{enumerate}

    \subsection{Definition: Beschränkte und alternierende Folgen}

    Sei $(a_n)_{n \in \IN}$ mit $a_n \in \IR \ \forall n \in \IN.$
    \begin{enumerate}[a)]
        \item $(a_n)$ heißt beschränkt $:\eqv \abs{a_n} \leq K$ für ein $K \geq 0$.
        \item $(a_n)$ heißt alternierend, falls die Folgenglieder abwechselnd positiv
        und negativ sind.
    \end{enumerate}

    \subsection{Beispiele}
    Aus 1.2):
    \begin{itemize}
        \item a, c, d, g) sind beschränkt
        \item b, e) sind unbeschränkt
        \item c) ist alternierend
    \end{itemize}
    
    \subsection{Definition: Konvergente Folgen}
    \begin{enumerate}[a)]
        \item Eine Folge $(a_n)_{n \in \IN}$ reeller Zahlen konvergiert gegen
        $a \in \IR$, wenn es zu jedem $\epsilon > 0$ ein $N \in \IN$ gibt (das von
        $\epsilon$ abhängig sein darf), so dass:

        $\abs{a_n-a} < \epsilon \quad \forall n \geq N$

        \textbf{Kurz: } $$\forall \epsilon > 0 \ \exists N \in \IN \ \forall n \geq N: \abs{a_n - a} < \epsilon$$
    
        \item $a \in \IR$ heißt Grenzwert oder Limes der Folge. Man schreibt: \\
        $\lim\limits_{n \to \infty}{a_n = a}$ oder $a_n \to a$ für $n \to \infty$ oder
        $a_n \xrightarrow[n \to \infty]{} a$ oder $a_n \to a$.

        \item Eine Folge $(a_n)$ mit Limes 0 heißt \underline{Nullfolge}.

        \item Eine Folge die nicht konvergent ist, heißt \underline{divergent}.
    \end{enumerate}

    \subsection{Bemerkung}

    $a_n \to a$ bedeutet anschaulich:
    Gibt man eine Fehlerschranke $\epsilon > 0$ vor, so sind ab einem bestimmten
    $N \in \IN$ alle Folgenglieder weniger als $\epsilon$ von a entfernt. Je kleiner
    $\epsilon$ gewählt wird, desto größer muss im allgemeinen $N$ gewählt werden.
    
    \begin{tikzpicture}
        \begin{axis}[
            sequence axis,
            ytick = {-1, 0, 1},
            yticklabels = {$a - \epsilon$, $a$, $a + \epsilon$},
            ymin = -1.5, ymax = 1.5,
            domain = 0:11,
            clip = false,
            width = 0.5\textwidth
        ]    
        \addplot[only marks] coordinates {
            (1, 1.1) (2, -1.2) (3, 0) (4, 1.3) (5, 0.9) (6, -0.4) (7, 0)
            (8, 0.1) (9, 0.2) (10, 0.1) (11, 0.2)
        };
        \addplot[name path = upper] {1};
        \addplot[name path = lower] {-1};
        \addplot[pattern = north east lines] fill between [of = lower and upper];
        
        \draw [dashed] (5, -1.5) -- (5, 1.5) node [right = 0pt, yshift = -5pt] {$N=5$};

        \draw [decorate, decoration = {brace, amplitude = 10pt, mirror, raise = 4pt}, yshift = 0pt] 
            (11, -1) -- (11, 1) node [right = 14pt, midway, align = left] {
                $\epsilon$ -- Umgebung von $a$/ \\
                $\epsilon$ -- Schlauch
            };
        \end{axis}
    \end{tikzpicture}

    Solch ein $N$ muss sich für jedes noch so kleine $\epsilon$ finden lassen.
    Ansonsten ist $(a_n)$ divergent.

    \subsection{Beispiele}

    \begin{enumerate}[a)]
        \item Behauptung: $a_n = \frac{1}{n}, (a_n)_{n \in \IN}$ ist Nullfolge

        Beweis:
        \begin{itemize}
            \item Wähle $\epsilon = \frac{1}{10}$. Dann ist für $N > 10$
           
            $$\abs{a_n-0} = \abs{\frac{1}{n}} = \frac{1}{n} \underset{N \geq n}{\leq} \frac{1}{N} \underset{N > 10}{<} \frac{1}{10} \quad \forall n \geq N$$
            \item Allgemein (beliebiges $\epsilon$)

            Sei $\epsilon > 0$. Dann ist für $N > \frac{1}{\epsilon}$

            $$\abs{a_n - 0} = \frac{1}{n} \underset{N \geq n}{\leq} \frac{1}{N} \underset{N > \frac{1}{\epsilon}}{<} \frac{1}{\frac{1}{\epsilon}} \quad \forall n \geq N$$
        \end{itemize}

        \item Behauptung: $(a_n)_{n \in \IN}$ mit $a_n = \frac{n + 1}{3n}$ hat Limes $a = \frac{1}{3}$.

        Beweis: Sei $\epsilon > 0$. Dann ist für $N \geq \frac{1}{3\epsilon}$

        $$\abs{a_n-n} = \abs{\frac{n+1}{3n}} = \frac{n + 1 - n}{3n} = \frac{1}{3n} \underset{N \geq n}{\leq} \boxed{\frac{1}{3N} < \epsilon} \quad \forall N \geq n$$
        
        \item $N$ muss nicht immer optimal gewählt werden.

        $$\frac{1}{n^3+n+5} \xrightarrow[n \to \infty]{} 0$$

        Sei $\epsilon > 0$, für $N > \frac{1}{\epsilon}$

        $$\abs{a_n - a} = \frac{1}{n^3+n+5} \underset{N \geq n}{\leq} \frac{1}{N^3+N+5} < \boxed{\frac{1}{N} < \epsilon}$$
    \end{enumerate}

\ifdefined\MAINDOC\else
\end{document}
\fi